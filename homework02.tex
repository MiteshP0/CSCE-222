\documentclass{article}
\usepackage{amsmath,amsthm,amsfonts,latexsym,paralist}

\theoremstyle{definition}
\newtheorem{problem}{Problem}
\newtheorem*{solution}{Solution}
\newtheorem*{resources}{Resources}

\newcommand{\names}[5]{
\begin{tabular}{|ll|}
\hline
\textbf{Name} & \textbf{Problems}\\
\hline
#1 & 1--10\\
#2 & #3\\
#4 & #5\\
\hline
\end{tabular}}


 




\begin{document}
\begin{center}
{\large
CSCE 222 [505] Discrete Structures for Computing\\[.5ex]
Fall 2015 -- Philip C. Ritchey\\}
\end{center}



\names{Mitesh Patel}{Malcolm Carr}{Helped with 1 2 3 5 10}{9-13-15}{}

\begin{resources} Discrete Mathematics and Its Application by Rosen (Chapter 2). http://math.stackexchange.com/questions/837318/if-a-function-has-a-inverse-that-is-well-defined-is-it-a-bijection. sharelatex.com and piazza.com %http://oeis.org/wiki/List_of_LaTeX_mathematical_symbols.%
\end{resources}

\bigskip

% set operations (even, odd, integers)
\begin{problem} (10 points) 
Supplementary Exercise 4
\end{problem}
\begin{solution}:
\newline
A) E $\cup$ O = set that contains either even or odd integers = set Z.
\newline
B) E $\cap$ O = set containing both E and O = EMPTY set since even and odd are total opposites.
\newline
C) Z - E = set of all integers excluding even integers = set O.
\newline
D) Z - O = set of all integers excluding odd integers = set E.
\end{solution}

\newpage

% subset and intersection
\begin{problem} (10 points) 
Supplementary Exercise 6
\end{problem}
\begin{solution}:
\newline
Given : A $\cap$ B = A which means that every element in set A also belongs in set B, and by definition of subset (A $\subseteq$ B)  which means that every element of A is also an element of B.\newline 
So: the quantification $\forall$x(x$\in$A$\to$x$\in$B) is True.
\end{solution}

\newpage

% size of intersection and size of union
\begin{problem} (10 points) 
Section 2.2, Exercise 46\\
\end{problem}
\begin{solution}:
\newline

Using : $|$ A $\cup$ B $|$ = $|$A$|$ + $|$B$|$ - $|$ A $\cup$ B $|$ (principle of inclusion exclusion)
\newline

$|$ A $\cup$ B $\cup$ C $|$ = $|$ A $\cup$ (B $\cup$ C) $|$ ------------ (associative law)
\newline
 = $|$A$|$ + $|$B $\cup$ C$|$ - $|$ A $\cap$ (B $\cup$ C) $|$ ------ (principle of inclusion exclusion)
\newline
 = $|$A$|$ + $|$B$|$ + $|$C$|$ - $|$ B $\cup$ C $|$ - $|$ (A$\cap$B) $\cup$ (A$\cap$C) $|$ ---------(distribution law)
 \newline
 = $|$A$|$ + $|$B$|$ + $|$C$|$ - $|$ B $\cup$ C $|$ - $|$ A $\cap$ B $|$ - $|$ A $\cap$ C $|$ + $|$ A $\cap$ B $\cap$ C $|$
 \newline
 = $|$ A $\cup$ B $\cup$ C $|$

\end{solution}

\newpage

% functions, size of range, size of domain
\begin{problem} (10 points) 
Supplementary Exercise 14
\end{problem}
\begin{solution}:
\newline
f(x) = x
\newline
x $\in$ S and f(x) $\in$ S
\newline
so x,y $\in$ S
\newline
if x=y then,
\newline
f(x)=f(y) therefore f is a one to one function, and
\newline
$|$ f(S) $|$ $\leq$ $|$ S $|$ is TRUE for all subsets of A.
\end{solution}

\newpage

% ceiling and floor (multiplication)
\begin{problem} (10 points) 
Supplementary Exercise 22
\end{problem}
\begin{solution}:
\newline
n $\in$ $\mathbb{Z}$
\bigskip
\newline
n=2y for some integer y

[$\dfrac{n}{2}$] [$\dfrac{n}{2}$] = [$\dfrac{2y}{2}$] [$\dfrac{2y}{2}$]
\bigskip
\newline
= $y^2$
\bigskip
\newline
[$\dfrac{n^2}{4}$] = [$\dfrac{4y^2}{4}$] = $y^2$
\bigskip
\newline
Therefore: 
[$\dfrac{n}{2}$] [$\dfrac{n}{2}$] = [$\dfrac{n^2}{4}$] = $y^2$
\end{solution}

\newpage

% bijection
\begin{problem} (10 points) 
Section 2.3, Exercise 22
\end{problem}
\begin{solution}:
\newline
A) y = -3x +4
\newline
x = -3y + 4
\newline
f$^{-1}$(x) = (x-4) / -3\newline
So it is a bijection since we can find the inverse. \bigskip
\newline
B)Not a bijection because it fails the horizontal line test, and by failing the horizontal line test it means the function is not one to one. We can also not that f(2) = f(-2) to confirm.
\bigskip
\newline
C) Not a bijection from $\mathbb{R}$ to $\mathbb{R}$ because at x = -2 the function can not divide by 0.
\bigskip
\newline
D) y = x$^5$ + 1
\newline
f$^{-1}$(x) = $\sqrt[5]{x-1}$
\newline
So this is a bijection because we can take the inverse.
\end{solution}

\newpage

% sequences, determine a rule
\begin{problem} (10 points) 
Supplementary Exercise 30
\end{problem}
\begin{solution}:
\newline
Last three terms add up to be the next term so:
\newline
Given:
\newline
a$_o$ = 1
\newline
a$_1$ = 3
\newline
a$_2$ = 4
\newline
Therefore the sequence is : a(n) = a$_{n-3}$ + a$_{n-2}$ + a$_{n-1}$ 
\newline
Next four terms : 
\newline
a$_8$ = a$_5$ + a$_6$ + $a_7$ = 27 + 50 + 92 = 169
\newline
a$_9$ = a$_6$ + a$_7$ + $a_8$ = 50 + 92 + 169 = 311
\newline
a$_{10}$ = a$_7$ + a$_8$ + $a_9$ = 92 + 169 + 311 = 572
\newline
a$_{11}$ = a$_8$ + a$_9$ + $a_{10}$ = 169 + 311 + 572 = 1052
\newline

\end{solution}

\newpage

% recurrence relations
\begin{problem} (10 points) 
Section 2.4, Exercise 22
\end{problem}
\begin{solution}:
\newline
A) a$_n$ = a$_{n-1}$ + 1000 + 0.05a$_{n-1}$
\newline
= 1.05a$_{n-1}$ + 1000
\newline
a$_o$ = 50000
\bigskip
\newline
B) a$_8$ = 1.05a$_7$ + 1000
\newline
a$_o$ = 50000
\newline
a$_1$ = 1.05(50000) + 1000 = 53500
\newline
a$_2$ = 1.05(53500) + 1000 = 57175
\newline
a$_3$ = 1.05(57175) + 1000 = 61033.8
\newline
a$_4$ = 1.05(61033.8) + 1000 = 65085.5
\newline
a$_5$ = 1.05(65085.5) + 1000 = 69339.8
\newline
a$_6$ = 1.05(69339.8) + 1000 = 73806.8
\newline
a$_7$ = 1.05(73806.8) + 1000 = 78497.1
\newline
a$_8$ = 1.05(78497.1) + 1000 = 83422
\bigskip
\newline
C) a$_n$ = 1.05$^n$ a$_o$ + $\Sigma$$_{i=0}$$^{n-1}$ 1.05$^i$ 1000
\newline
= 1.05$^n$ 50000 + 1000 $\Sigma$$_{i=0}$$^{n-1}$ 1.05$^i$ 
= 1.05$^n$ 50000 + 1000 (20 1.05$^n$ - 20)
\newline
= 70000 (1.05$^n$) - 20000
\end{solution}

\newpage

% telescoping, derive summation formula
\begin{problem} (10 points) 
Section 2.4, Exercise 38
\end{problem}
\begin{solution}:
\newline
a$_k$ = $k^3$ 
\newline
telescoping: a$_k$ - $a_{k-1}$ = $k^3$-$(k-1)^3$
\newline
expanding$(k-1)^3$ we then get: 
\newline
= $k^3$ - ($k^3$-$1^3$-$3k^2$+3k)
\newline
= 1 + 3$k^2$ - 3k
\newline
$k^2$ = (3k-1) / 3
\newline
$k^2$ = ( $k^3$ - $(k-1)^3$ + 3k - 1 ) / 3
\newline
$\Sigma_{k=1}^n k^2$ = (1/3) $\Sigma_{k=1}^n$ ($k^3 - (k-1)^3 + 3k - 1$)
\newline
= (1/3) $\Sigma_{k=1}^n$ ($k^3 - (k-1)^3$) + 3 $\Sigma_{k=1}^n$ k - $\Sigma_{k=1}^n$ 1
\newline
by formula: 
\newline
$\Sigma_{k=1}^n$ k = n(n+1) / 2
\newline
$\Sigma_{k=1}^n$ 1 = n 
\newline
by telescoping: 
$\Sigma_{k=1}^n$ ($k^3 - (k-1)^3$) = $n^3 + 0$
\newline
by substituting:
$\Sigma_{k=1}^n$ $k^2$ = (1/3)[$n^3$ - 0 + 3 (n(n+1) / 2) - n]
\newline
= (1/3)[$n^3$ + ( (3$n^2$ + n) / 2 )]
\newline
= n (n+1)(2n+1)  / 6 
\end{solution}

\newpage

% identity matrix
\begin{problem} (10 points) 
Supplementary Exercise 38
\end{problem}
\begin{solution}:
\newline
\textbf{A}=c\textbf{I}
\newline
We are trying to show that \textbf{AB}=\textbf{BA}.
\newline
Therefore, lets give values to c and the matrix I, and B. 
\newline
c = 2
\newline
\textbf{I} = $\begin{bmatrix}
    1 & 2 \\
    3 & 4 \\
\end{bmatrix}$
\newline
\textbf{B} = $\begin{bmatrix}
    4 & 3 \\
    2 & 1 \\
\end{bmatrix}$
\newline
So: \textbf{A} = 2  $\begin{bmatrix}
    1 & 2 \\
    3 & 4 \\
\end{bmatrix}$ = $\begin{bmatrix}
   2 & 4 \\
    6 & 8 \\
\end{bmatrix}$
\newline
AB = $\begin{bmatrix}
    2 & 4 \\
    6 & 8 \\
\end{bmatrix}$ $\begin{bmatrix}
    4 & 3 \\
    2 & 1 \\
\end{bmatrix}$ = $\begin{bmatrix}
    8 & 12 \\
    12 & 8 \\
\end{bmatrix}$
\newline
BA = $\begin{bmatrix}
    4 & 3 \\
    2 & 1 \\
\end{bmatrix}$ $\begin{bmatrix}
    2 & 4 \\
    6 & 8 \\
\end{bmatrix}$ = $\begin{bmatrix}
    8 & 12 \\
    12 & 8 \\
\end{bmatrix}$
\newline
Therefore we can conclude that \textbf{AB} = \textbf{BA}

\end{solution}



\end{document}
