\documentclass{article}
\usepackage{amsmath,amsthm,latexsym,paralist}
\usepackage{algorithm}
\usepackage{algpseudocode}
\usepackage[margin=1in]{geometry}

\theoremstyle{definition}
\newtheorem{problem}{Problem}
\newtheorem*{solution}{Solution}
\newtheorem*{resources}{Resources}

\newcommand{\names}[5]{
\begin{center}
\begin{tabular}{|ll|}
\hline
\textbf{Name}  & \textbf{Problems}\\
\hline
#1 & 1--10\\
#2 & #3\\
#4 & #5\\
\hline
\end{tabular}
\end{center}
}
\newcommand{\honor}{\noindent \textbf{Aggie Honor Statement: }On my honor, as an Aggie, I have neither
  given nor received any unauthorized aid on any portion of the
  academic work included in this assignment. Furthermore, I have
  disclosed all resources (people, books, web sites, etc.) that have
  been used to prepare this homework. 
}

 
\newcommand{\checklist}{\noindent\textbf{Checklist:}
\begin{compactenum}
\item Did you type your full name and UIN and those of any collaborators? 
\item Did you abide by the Aggie Honor Code?
\item Did you solve all problems and start a new page for each? 
\item Did you submit
\begin{compactenum}
\item your \LaTeX\ source file?
\item your  PDF file?
\end{compactenum}
\end{compactenum}
}

\newcommand{\problemset}[1]{\begin{center}\textbf{Problem Set #1}\end{center}}
\newcommand{\duedate}[1]{\begin{quote}\textbf{Due dates:} Electronic
    submission of \LaTeX\ and PDF files of this homework is due on
    \textbf{#1} on eCampus (\texttt{http://ecampus.tamu.edu}). 
    \end{quote}}


\begin{document}
\begin{center}
{\large
CSCE 222 [505] Discrete Structures for Computing\\[.5ex]
Fall 2015 -- Philip C. Ritchey\\}
\end{center}

\problemset{6}

\duedate{21 October 2015 (Wednesday) before 11:30 a.m.}

\names{Mitesh Patel}
{10-20-15}{}
{UIN: 124002210}{}

\begin{resources} www.piazza.com \\
https://en.wikipedia.org/wiki/Knapsack\_problem\\
http://www.cplusplus.com/forum/beginner/69889/ \\
https://users.cs.duke.edu/~ola/ap/recurrence.html \\
https://proofwiki.org/wiki/Inclusion-Exclusion\_Principle


\end{resources}

\bigskip

% ternary strings with consecutive 0s or 1s
\begin{problem} (15 points) Section 8.1, Exercise 16
\end{problem}
\begin{solution} :
\\
A) We know that 0,1,2 can be in a ternary string.
\\
Correct way: We can append these to a string with length n-1. Therefore we would get $3a_{n-1}$
\\
Incorrect strings of length n-2 can be: $3^{n-2} - a_{n-2}$
\\
We can add 00 or 11 to this string to make it a valid string: so: $2(3^{n-2}-a_{n-2})$
\\
Therefore, $a_n = 3a_{n-1}+2(3^{n-2}) - 2a_{n-2}$
\\

B) initial conditions:
\\
$a_o = 0$
\\
$a_1 = 0$
\\

C) $a_2 = 3a_1+2(3^0) - 2a_o = 3(0)+2(1)-2(0) = 2$
\\
$a_3 = 3a_2+2(3^1) - 2a_1 = 3(2)+2(3)-2(0) = 12$
\\
$a_4 = 3a_3+2(3^2) - 2a_2 = 3(12)+2(9)-2(2) = 50$
\\
$a_5 = 3a_4+2(3^3) - 2a_3 = 3(50)+2(27)-2(12) = 180$
\\
$a_6 = 3a_5+2(3^4) - 2a_4 = 3(180)+2(81)-2(50) = 602$

\end{solution}

\newpage

% dynamic programming
\begin{problem} (15 points) Section 8, Supplementary Exercise 14
\end{problem}
\begin{solution} :
\\
A) Since $w_j > w$
\\
Weight of item j is greater than w. Using M(j,w) it means we cant have $w_j > w$
\\
So: M(j,w) = M(j-1,w) - maximum weight in j items that don't exceed w
\\

B) Consider $w_j \leq w$
\\
Using item j:
\\
the max weight that is subset of first j-1 items is M(j-1,w-$w_j$)
\\
therefore, M(j,w) = max(M(j-1,w),$w_j$ + M(j-1,w-$w_j$))
\\

C)
\begin{algorithm}
\caption{dynamic programming}
\label{}
\begin{algorithmic}[1]
\Procedure{maxweight}{$w_1 .... w_n 0 \leq w \leq W$}
    \For { w := 0 to W }
    \State M(0,W) = 0
    \EndFor
    \For {j := 1 to n}
    \For{w:=0 to W}
         \If {$W_j \leq W$ } 
        M(j,w) = max(M(j-1,w),$w_j$ + M(j-1,w-$w_j$))
        \Else \State Return $large$
        \EndIf
        \EndFor
        \EndFor
\EndProcedure
\end{algorithmic}
\end{algorithm}



D) We can use another function to store the values and set a temp variable to hold fixed weight limit. Then we can subtract this temp weight by item j.

\end{solution}

\newpage

% goats on an island
\begin{problem} (15 points) Section 8.2, Exercise 46
\end{problem}
\begin{solution} :
\\

A) $a_o$ = 2
\\
$a_n$ = $2a_{n-1} + 100$
\\

B) $r^n = 2r^{n-1}$
\\
$r^n / r^n r^{-1}  = 2$
\\
r = 2
\\
$a_n^{(h)}$ = $\alpha$$r^n = \alpha2^n$
\\
$a_n$ = $\alpha2^n - 100$
\\
$a_o$ = 2 = $\alpha2^0 - 100$
\\
$\alpha = 102$
\\
$a_n = 102(2^n)-100$
\\

C) $a_n = 2a_{n-1}-n$
\\

D) $a_n = 2a_{n-1}-n$
\\
$a_n = a_n^{(h)} + a_n^{(p)}$
\\
first lets solve its associated linear homogeneous 
\\
$a_n = 2a_{n-1}$
\\
r = 2
\\
 $a_n^{(h)}$ = $\alpha2^n$
 \\
 F(n) = n
 \\
 suppose p(n) = cn + d is such a solution
 \\
 cn + d = 2(C(n-1)+d) - n
 \\
 = 2(cn-c+d)-n
 \\
 = 2cn-2c+2d-n
 \\
 cn+n-2cn = -2c+2d-d
 \\
 n(c+1-2c) = -2c + d
 \\
 (-c+1)n + (2c+d) = 0
 \\
 c = 1 , and d = -2
 \\
 so plugging in:
 \\
 $a_n^{(p)} = n-2_n = n - 2 + \alpha2^n$
 \\
$ a_o = 2 = -2 + \alpha$
\\
$\alpha $= 4
\\
Therefore, $a_n = n - 2 + 4(2^n)$
  

\end{solution}

\newpage

% Develop and solve recurrence relation (requires solving 14)
\begin{problem} (15 points) Section 8.3, Exercise 16
\end{problem}
\begin{solution} :
\\
from 14: number of rounds from two sets is f(n/2), requires one more round to get winner.
\\
therefore recurrence relation would be f(n) = f(n/2) + 1, f(1) = 0 where n = $2^k$
\\

Now 16:
\\
$a_{(2^{k})}$ = $a_{(2^{k-1})}$ + 1\\
= $a_{(2^{k-1})}$ + 1\\
= $a_{(2^{k-2})}$ + 1 + 1\\
= $a_{(2^{k-3})}$ + 1 + 1 + 1\\
.
\\
.\\
.\\
= $a_{(2^{k-i})}$ + i
\\
$i_{max}$ = k
\\
so : $a_{(2^{k-k})}$ + k
\\
= $a_1$ + k
\\
from 14 : = 0 + k = k = $a_{2^n}$
 
\end{solution}

\newpage

% generating functions and counting
\begin{problem} (15 points) Section 8.4, Exercise 14
\end{problem}
\begin{solution} :
\\

$0\leq x \leq 3$\\
$x_1 + x_2 + x_3 + x_4 + x_5 = 12 $\\
$(1+x+x^2+x^3)^5$ = $(1-x^4)^5$ / $(1-x)^5$\\
by expanding first couple of terms = $\frac{1 - 5x^4 + 10x^8 - 10x^{12} +...}{(1-x)^5}$\\
using: $\frac{1}{1-(ax)^n}$ = $\Sigma_{k=0}^{\infty}$ C(n+k-1,k)$a^k x^k$\\
n = 12, k = 4 \\ 
n = 8, k = 4 \\ 
n = 4, k = 4 \\ 
n = 0, k = 4 \\ 

$n+k-1 \choose k$\\

$16 \choose 4$ - 5$12 \choose 4$ + 10$8 \choose 4$ - 10$1 \choose 1$ \\

= 1820 - 2475 + 700 -10 = 35
 
\end{solution}

\newpage

% inclusion exclusion
\begin{problem} (15 points) Section 8.5, Exercise 14
\end{problem}
\begin{solution} : 
\\
26! permutations total \\ 
$|$A$|$ := set that contains "fish"\\
$|$B$|$ := set that contains "rat"\\
$|$C$|$ := set that contains "bird"\\ 
\\
so: \\
$|$A$|$ = 26 - 4 + 1 = 23! // we add one because the string is one block
\\
$|$B$|$ = 26 - 3 +1 = 24!
\\
$|$C$|$ = 26 - 4 + 1 = 23! \\

$| A \cup B \cup C | = |A| + |B| + |C| - |A\cap B| - |A \cap C| - |B \cap C| + |A\cap B \cap C| $ \\

$|A \cap B| = 21!$ because fish and rat are 7 diff letters so 19 letters left and they are two blocks of string.
\\
$|A\cup B|$ = empty set because repeated i letter
\\
$|B \cap C|$ = empty set because repeated r letter therefore , $|A\cap B \cap C|$  is empty
\\

$| A \cup B \cup C | $ = 24! + 23! + 23! -21!
\\
finally subtract 26! from above so:

26! - 24! - 23! - 23! + 21!  //  ----- answer
\end{solution}

\newpage

% inclusion exclusion
\begin{problem} (10 points) Section 8.6, Exercise 2
\end{problem}
\begin{solution}:
\\
$P_1$ := property for altitude sickness \\
$P_2$ := property for not in shape \\
$P_3$ := property for getting allergies \\

$N(P_1^{'}P_2^{'}P_3^{'})$ = $N - N(P_1) - N(P_2) - N(P_3) + N(P_1P_2) + N(P_2P_3) + N(P_1P_3) - N(P_1P_2P_3)$
\\

N = 1000 \\
$N(P_1)$ = 450\\
$N(P_2)$ = 622\\
$N(P_3)$ = 30 \\
$N(P_1P_2)$ = 111 (altitude sickness and not in shape)\\
$N(P_2P_3)$ = 14 (not in shape and allergies )\\
$N(P_1P_3)$ = 18 (altitude sickness and allergies)\\
$N(P_1P_2P_3)$ = 9 (all three)\\

Plugging in:
\\
1000 - 450 - 622 - 30 + 111 + 14 + 18 - 9 = 32
\end{solution}

%\newpage


\goodbreak
\noindent
\textbf{Wildcard Quiz Problems} (the quiz on Friday could also be one of these)\\
Section 8.1, Exercise 12\\
Section 8.2, Exercise 6\\
Section 8.3, Exercise 22\\
Section 8.3, Exercise 13\\



\goodbreak
\honor

\bigskip
\checklist
\end{document}
