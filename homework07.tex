\documentclass{article}
\usepackage{amsmath,amsthm,latexsym,paralist}
\usepackage[margin=1in]{geometry}

\theoremstyle{definition}
\newtheorem{problem}{Problem}
\newtheorem*{solution}{Solution}
\newtheorem*{resources}{Resources}

\newcommand{\names}[5]{
\begin{center}
\begin{tabular}{|ll|}
\hline
\textbf{Name}  & \textbf{Problems}\\
\hline
#1 & 1--10\\
#2 & #3\\
#4 & #5\\
\hline
\end{tabular}
\end{center}
}
\newcommand{\honor}{\noindent \textbf{Aggie Honor Statement: }On my honor, as an Aggie, I have neither
  given nor received any unauthorized aid on any portion of the
  academic work included in this assignment. Furthermore, I have
  disclosed all resources (people, books, web sites, etc.) that have
  been used to prepare this homework. 
}

 
\newcommand{\checklist}{\noindent\textbf{Checklist:}
\begin{compactenum}
\item Did you type your full name and that of all collaborators? 
\item Did you abide by the Aggie Honor Code?
\item Did you solve all problems and start a new page for each? 
\item Did you submit
\begin{compactenum}
\item your \LaTeX\ source file?
\item your  PDF file?
\end{compactenum}
\end{compactenum}
}

\newcommand{\problemset}[1]{\begin{center}\textbf{Problem Set #1}\end{center}}
\newcommand{\duedate}[1]{\begin{quote}\textbf{Due dates:} Electronic
    submission of \LaTeX\ and PDF files of this homework is due on
    \textbf{#1} on eCampus (\texttt{http://ecampus.tamu.edu}). 
    \end{quote}}


\begin{document}
\begin{center}
{\large
CSCE 222 [505] Discrete Structures for Computing\\[.5ex]
Fall 2015 -- Philip C. Ritchey\\}
\end{center}

\problemset{7}

\duedate{30 October 2015 (Friday) before 11:30 a.m.}

\names{Mitesh Patel}
{UIN: 124002210}{}
{10-24-15}{}

\begin{resources} 
http://www.inf.ed.ac.uk/teaching/courses/dmmr/slides/13-14/Ch2.pdf\\
http://web.ift.uib.no/Teori/KURS/WRK/TeX/symALL.html\\
http://math.stackexchange.com/questions/1496817/relation-in-diagram-is-it-reflexive-symmetric-transitive-and-antisymmetric\\
http://www.math.cornell.edu/~levine/18.312/alg-comb-lecture-7.pdf

\end{resources}

\bigskip


% enumerating relations (includes problem 42)
\begin{problem} (15 points) Section 9.1, Exercise 44 (a,c,d,f)
\end{problem}
\begin{solution} :
\\
16 different relations on set \{0,1\}
\\
AxA = \{(0,0),(0,1),(1,0),(1,1)\}
\\

R$_1$ = $\emptyset$
\\
R$_2$ = \{(0,0)\}\\
R$_3$ = \{(0,1)\}\\
R$_4$ = \{(1,0)\}\\
R$_5$ = \{(1,1)\}\\
R$_6$ = \{(0,0),(0,1)\}\\
R$_7$ = \{(0,0),(1,0)\}\\
R$_8$ = \{(0,0),(1,1)\}\\
R$_9$ = \{(0,1),(1,0)\}\\
R$_{10}$ = \{(0,1),(1,1)\}\\
R$_{11}$ = \{(1,0),(1,1)\}\\
R$_{12}$ = \{(0,0),(0,1),(1,0)\}\\
R$_{13}$ = \{(0,0),(0,1),(1,1)\}\\
R$_{14}$ = \{(0,0),(1,0),(1,1)\}\\
R$_{15}$ = \{(0,1),(1,0),(1,1)\}\\
R$_{16}$ = \{(0,0),(0,1),(1,0),(1,1)\}\\
\\

A) $R_8, R_{13}, R_{14}, R_{16},$ because contain all pairs in form of (a,a)\\

C)$R_1, R_2, R_5, R_8, R_9, R_{12}, R_{15}, R_{16}$ symmetric \\

D) $R_1$ through $R_{14}$ anti symmetric because no pair of elements a and b with a != b such that both (a,b) and (b,a) belong to the relation\\

F) $R_1$ through $R_{14}$ and $R_{16}$ are transitive by showing if (a,b) and (b,c) belong to the relation then (a,c) does.

\end{solution}

\newpage

% Constructing relations with given properties
\begin{problem} (15 points) Section 9, Supplementary Exercise 2
\end{problem}
\begin{solution} :
\\

A) R = \{(a,a),(b,b),(c,c),(d,d) , (a,b)(b,a)(c,b)(b,c)\} \\
reflexive since our pairs include all ordered pairs such that (x,x) $\in$ R \\
Symmetric since (b,a) $\in$ R whenever (a,b) $\in$ R and (c,d) $\in$ R whenever (b,c) $\in$ R \\
Not Transitive since (a,b) $\in$ R and (b,c) $\in$ R, but (a,c) $\notin$ R \\

B) R = $\emptyset$ since nothing in this set, nothing is related. \\

C) R = \{(a,b),(b,c)\}\\
irreflexive because (a,a) $\in$ R \\
anti symmetric since (b,a) $\in$ R and False implying True is always True therefore antisymmetric \\
Not transitive since (a,c) $\notin$ R \\

D) R = \{(a,a),(b,b),(c,c),(d,d),(a,b),(b,a),(c,a),(b,c)\}
\\
reflexive since all x (x,x) $\in$ R
\\
not symmetric since (b,c) $\in$ R but not (c,b)
\\
not antisymmetric since a != b\\
transitive since (a,b) $\in$ R and (b,c) $\in$ R then (a,c) $\in$ R is true \\

E) R = {(a,c),(b,a),(c,c),(a,c)} \\
(a,a) $\in$ R so not reflexive and (c,c) $\in$ R so irreflexive \\
Not symmetric since (a,c) $\in$ R but (c,a) $\notin$ R. Not anti symmetric because a != b\\
Not transitive since (b,a) $\in$ R and (a,c) $\in$ R then (b,c) $\notin$ R

\end{solution}

\newpage

% Reasoning about relations
\begin{problem} (15 points) Section 9, Supplementary Exercise 8
\end{problem}
\begin{solution} :
\\
Yes, by proof by contradiction,
\\
let A = \{(a,b)\}
\\
Constructing a symmetric relation: R = \{(b,a),(a,b)\}
\\

Let (c,d) $\in$  \~{R} then (c,d) $\notin$ R
\\
if (d,c) $\in$ R it would be contradiction and not symmetric
\\
so (d,c) $\in$ \~{R}
\\
therefore,  (d,c) $\in$ \~{R}
\\
finally: \~{R} = \{(c,d),(d,c)\} which is symmetric

\end{solution}

\newpage

% Representing relations
\begin{problem} (15 points) Section 9.3, Exercise 34
\end{problem}
\begin{solution} :
\\
Relation R on a set A can be a graph that has values of A as the vertices and ordered pairs (a,b) as edges \\

Since (a,b) $\in$ R iff (a,b) $\notin$ \~{R} when there is an edge from a to b in graph of R, then the edge isnt drawn in \~{R} \\

So: digraph of \~{R} is made by edges that aren't in digraph of R
\end{solution}

\newpage

% Equivalence relations
\begin{problem} (15 points) Section 9, Supplementary Exercise 20
\end{problem}
\begin{solution} :
\\
A) x can have same zodiac sign with x so (x,x) $\in$ R - reflexive
\\
x and y can have same zodiac sign and therefore y and x would have same sign so (x,y) $\in$ R $\implies$ (y,x) $\in$ R is true. So it is symmetric.
\\
(x,y) $\in$ R, and (y,x) $\in$ R, x and y have same sign so y and z have same sign. therefore (x,z) $\in$ R so it is transitive. So it is a equivalence relation.
\\

B) x can have same birth year as x so (x,x) $\in$ R - reflexive
\\
x and y can have same birth year and therefore y and x would have same birth year so (x,y) $\in$ R $\implies$ (y,x) $\in$ R is true. So it is symmetric.
\\
 (x,y) $\in$ R, and (y,x) $\in$ R, x and y have same birth year, then y and z have same birth year, therefore (x,z) $\in$ R so it is transitive. So it is a equivalence relation.
 \\
 
 C) x and x have been in same city so (x,x) $\in$ R is reflexive
 \\
 x and y have been in same city so y and x have been in same city therefore (x,y) $\in$ R $\implies$ (y,x) $\in$ R is true so symmetric.
 \\
 (x,y) $\in$ R, and (y,x) $\in$ R, x and y have been in same city, but it is not valid that x and z have been in same city, therefore (x,z) $\notin$ R so it is NOT transitive. So it is NOT a equivalence relation.
\end{solution}

\newpage

% Posets
\begin{problem} (15 points) Section 9.6, Exercise 6
\end{problem}
\begin{solution}:
\\
A) a = a for every real number, so it is reflexive \\
(a=a $\wedge$ b=a) $\implies$ (a=b) is true so anti symmetric \\
(a=b $\wedge$ b=c) $\implies$ (a=c) is true so transitive \\
therefore it (R, =) is poset.
\\

B) a $<$ a  - NO this is not true, so it is not reflexive so just by checking this case we can conclude (R, $<$) not a poset. \\

C) a$\leq$a - yes for every a it will be = so true so reflexive \\ 
(a $\leq$ b $\wedge$ b $\leq$ a) $\implies$ (a=b) is true so anti-symmetric \\ 
(a $\leq$ b $\wedge$ b $\leq$c) $\implies$ (a$\leq$c) is true so transitive, \\
therefore (R,$\leq$) is poset.\\

D) a =a for all real number so it is not reflexive \\
therefore (R,$\neq$) is not a poset because a=a for all a. 
\end{solution}

\newpage

% Composite key
\begin{problem} (10 points) Section 9.2, Exercise 6
\end{problem}
\begin{solution} :
\\
Professor and course number.
\\
Professor and time.
\\

This is because since no two professors with same name have same course number or time assessment.
\end{solution}

%\newpage

\goodbreak
\noindent
\textbf{Wildcard Quiz Problems} (the quiz on Friday could also be one of these)\\
Section 9.1, Exercise 2\\
Section 9.2, Exercise 2\\
Section 9.3, Exercise 18/20/22\\
Section 9.5, Exercise 2\\
Section 9.6, Exercise 6\\


\goodbreak
\honor

\bigskip
\checklist
\end{document}
