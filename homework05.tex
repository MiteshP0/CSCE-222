\documentclass{article}
\usepackage{amsmath,amsthm,latexsym,paralist}
\usepackage{algorithm}
\usepackage{algpseudocode}
\usepackage{fancybox}
\usepackage{forest}
\usetikzlibrary{positioning,arrows.meta}

\usepackage[margin=1in]{geometry}

\theoremstyle{definition}
\newtheorem{problem}{Problem}
\newtheorem*{solution}{Solution}
\newtheorem*{resources}{Resources}

\newcommand{\names}[5]{
\begin{center}
\begin{tabular}{|ll|}
\hline
\textbf{Name}  & \textbf{Problems}\\
\hline
#1 & 1--10\\
#2 & #3\\
#4 & #5\\
\hline
\end{tabular}
\end{center}
}
\newcommand{\honor}{\noindent \textbf{Aggie Honor Statement: }On my honor, as an Aggie, I have neither
  given nor received any unauthorized aid on any portion of the
  academic work included in this assignment. Furthermore, I have
  disclosed all resources (people, books, web sites, etc.) that have
  been used to prepare this homework. 
}

 



\newcommand{\problemset}[1]{\begin{center}\textbf{Problem Set #1}\end{center}}
\newcommand{\duedate}[1]{\begin{quote}\textbf{Due dates:} Electronic
    submission of \LaTeX\ and PDF files of this homework is due on
    \textbf{#1} on eCampus (\texttt{http://ecampus.tamu.edu}). 
    \end{quote}}


\begin{document}
\begin{center}
{\large
CSCE 222 [505] Discrete Structures for Computing\\[.5ex]
Fall 2015 -- Philip C. Ritchey\\}
\end{center}

\problemset{5}

\duedate{14 October 2015 (Wednesday) before 11:30 a.m.}

\names{Mitesh Patel}
{10-10-15}{}
{UIN:124002210}{}

\begin{resources} Discrete Mathematics and Its Applications Chapter 6 \\
http://www.texample.net/tikz/examples/feature/trees/ \\
http://tex.stackexchange.com/questions/5447/how-can-i-draw-simple-trees-in-latex \\
http://www.mycstutorials.com/articles/sorting/quicksort \\
https://en.wikipedia.org/wiki/Merge-sort\\
http://tex.stackexchange.com/questions/213770/how-to-combine-a-top-down-and-bottom-up-binary-tree-in-one-picture

\end{resources}

\bigskip

% recursive algorithm for sum 1..n
\begin{problem} (10 points) Section 5.4, Exercise 8
\end{problem}
\begin{solution}:
\\
\begin{algorithm}
\caption{Recursive algorithm for finding sum of first n positive integers}
\label{}
\begin{algorithmic}[1]
\Procedure{sum}{n : integer}
         \If {n == 1} 
        return 1
        \Else \State return n + sum(n-1)
        \EndIf
        
\EndProcedure
\end{algorithmic}
\end{algorithm}
\end{solution}

\newpage

% prove that above algorithm is correct
\begin{problem} (10 points) Section 5.4, Exercise 16
\end{problem}
\begin{solution}:
\\
Proof by Mathematical Induction sum(n) = n + sum(n-1) \bigskip
\\
Basis: \\ If n=1 then the first step of the algorithm tells us that 1 + sum(1-1) = 1, and this is true because the sum of 1 positive integer is just 1, therefore our basis step checks out.\bigskip
\\
Inductive:
\\
Assume: sum(k) = k + sum(k-1) , algorithm correctly computes first k positive integers.
\\
Show: sum(k+1) = k + 1 + sum(k+1-1) , algoritm correctly computes first k + 1 integers.
\\
\hspace*{2.6cm} = k + 1 + sum(k)
\\
Using our IH we can conclude that our above statement is true for k+1. Therefore by PMI the algorithm is correct.
\end{solution}

\newpage

% do mergesort
\begin{problem} (10 points) Section 5.4, Exercise 44
\end{problem}
\begin{solution}:
\\

\newsavebox\Downtree
\newsavebox\Uptree

\tikzset{
  nleft/.style={text width=15pt,midway,left,font=\strut\scriptsize},
  nright/.style={text width=15pt,midway,right,font=\strut\scriptsize},
  }

\savebox\Downtree{\begin{forest}
for tree={
  s sep=22pt,
  l sep=20pt,
  where n children=0{inner ysep=0pt}{draw,circle},
  edge={->,>=latex}
}
[4 3 2 5 1 6 7 6 
  [4 3 2 5
  [4 3
  [4
  ]
  [3
  ]
  ]
  [2 5
  [2
  ]
  [5
  ]
  ]
  ]
  [1 8 7 6 
  [1 8 
  [1
  ]
  [8
  ]
  ]
  [7 6
  [7
  ]
  [6
  ]
  ]
  ]
  ]
]
\end{forest}%
}
\savebox\Uptree{\begin{forest}
for tree={
  grow'=north,
  s sep=25pt,
  l sep=20pt,
  where n children=0{inner ysep=0pt}{draw,circle},
  edge={->,>=latex}
}
[1 2 3 4 5 6 7 8 
  [2 3 4 5
    [3 4
    [4
    ]
    [3
    ] 
    ]
    [2 5
    [2
    ] 
    [5
    ]
    ]
  ]
  [1 6 7 8
    [1 8
    [1
    ]
    [8
    ]
    ]
    [6 7
    [7
    ]
    [6
    ]
    ]
    ]
  ]
]
\end{forest}%
}



\begin{tikzpicture}[>={Latex[open]}]
\node[inner sep=0pt] (Down) {\usebox\Downtree};
\coordinate (aux);
\node[inner sep=0pt,below=0pt of Down] (Up) {\usebox\Uptree};


\end{tikzpicture}


\end{solution}

\newpage

% do quicksort
\begin{problem} (10 points) Section 5.4, Exercise 50
\end{problem}
\begin{solution}:
\\
3,5,7,8,1,9,2,4,6
\\ 
The median is 5 so that means out pivot is 5.
\\

Now just looking at the first half of the numbers: 3,1,2,4 - our pivot is 3 therefore the numbers less than 3 go to the left of three and greater than 3 go to the right of 3. We get 1,2,3,4 as a result.
\\

Now looking at the other half of the numbers: 7,8,9,6 - our pivot is 7 and using the same method above where the numbers less than 7 go to the left and the numbers greater go to the right. We get 6,7,8,9 as a result.
\\

Therefore, as a result we get [ 1,2,3,4 ], 5 , [6,7,8,9] = 1,2,3,4,5,6,7,8,9 sorted.
\end{solution}

\newpage

% basics of counting
\begin{problem} (10 points) Section 6, Supplementary Exercise 8
\end{problem}
\begin{solution}:
\\
   A) 
\\
100 through 999 is a 3 digit positive integer
\\
so: (1000-100) = 900
\\
900 positive integers with three decimal digits (100 through 900)
\\

B) 
\\We can see that there are 9 ways for one positive integer with one decimal digit (1 through 9)
\\
And from part A the number of positive 3 digit integers is 900
\\
so: 900 + 9 = 909 numbers that have odd number of decimal digits
\\

C)\\
The possibilities are:
\\
 A one digit number that is 9 = 1 
 \\
 A two digit number with the number 9 at the tens place and 0-9 at ones place is = 1 + 9 = 10
 \\
 A two digit number with 9 at ones place and 1-8 at tens place = 1 + 7 = 8
 \\
 A three digit number where 0-9 in the ones place and tens place, and 9 in 100s place = 10(10) = 100
 \\
 A three digit number where 9 is in tens place and 1-8 in 100s place and 0-8 in ones place = 9(10) = 90
 \\
 A three digit number where 9 in ones place, 0-8 in tens, and 1-8 in 100s = 9(9) = 81
 \\
 Therefore by adding them all up we get 1+10+8+100+90+81 = 290 numbers have at least one decimal digit equal to 9
\\

D) 
\\
1-9 on ones place - 1,2,3,4,5,6,7,8,9 - we have 4 even numbers
\\
0-9 in tens place - 0,1,2,3,4,5,6,7,8,9 - we have 5 even numbers (counting 0)
\\
0-9 in 100s place - - 0,1,2,3,4,5,6,7,8,9 - we have 5 even numbers (counting 0)
\\
Therefore, 4(5$^2$) = 120 numbers have no odd decimal digits.
\\

E)\\
We can have 55 a two digit number where theres only one way in which the ones place has to be 0 - 1
\\
We can have \_ 55 a three digit number where 0-9 is in the blank so 10 ways - 10
\\
We can have 55 \_ a three digit number where 0-9 except 5 and 0 in the blank so 8 ways - 8
\\
Therefore there are 1+10+8 = 19 numbers that have consecutive digits equal to 5
\\

F)\\
There are 9 three digit palindromes : 111, 222, .... , 999
\\
There are 9 two digit palindromes : 11, 22, ... , 99
\\
3 digit palindromes could also be where middle value is numbers 0-9: 1 \_ 1, 2 \_ 2, ...., 9 \_ 9 . Where the middle value can be a number from 0-9, so 10 ways to do this on 9 potential palindromes = 9(10) = 90
\\
Therefore summing them up we get: 90+9+9 = 108 numbers that are palindromes.
\end{solution}

\newpage

% pigeon hole principle
\begin{problem} (10 points) Section 6, Supplementary Exercise 12
\end{problem}
\begin{solution}:
\\
There are 7 days in a week, and 12 months.
\\
Using this we can multiple 12 by 7 which gives us 84. Now using the pigeon hole principle this gives us:
\\
k = 84
\\
k+1 = 84+1 = 85 people where at least two would be born on the same day of the week and on the same month.
\end{solution}

\newpage

% permutations and combinations
\begin{problem} (10 points) Section 6.3, Exercise 30
\end{problem}
\begin{solution}:
\\
A)
\\
We have 7 women and 9 men. 
\\
Using C(n,r) = $\frac{n!}{r!(n-r)!}$
\\
C(7,1)C(9,4) + C(7,2)C(9,3) + C(7,3)C(9,2) + C(7,4)C(9,1) + C(7,5)C(9,0) =
\\
$\frac{7!}{1!(6!)}$$\frac{9!}{4!(5!)}$ + $\frac{7!}{2!(5!)}$$\frac{9!}{3!(6!)}$ + $\frac{7!}{3!(4!)}$$\frac{9!}{2!(7!)}$ + $\frac{7!}{4!(3!)}$$\frac{9!}{1!(8!)}$ + $\frac{7!}{5!(3!)}$$\frac{9!}{0!(9!)}$ = 7(126) + 21(84) + 35(36) + 35(9) + 21(1)
\\
= 4242 ways
\\

B)
\\
C(7,1)C(9,4) + C(7,2)C(9,3) + C(7,3)C(9,2) + C(7,4)C(9,1) =
\\
$\frac{7!}{1!(6!)}$$\frac{9!}{4!(5!)}$ + $\frac{7!}{2!(5!)}$$\frac{9!}{3!(6!)}$ + $\frac{7!}{3!(4!)}$$\frac{9!}{2!(7!)}$ + $\frac{7!}{4!(3!)}$$\frac{9!}{1!(8!)}$ = 7(126) + 21(84) + 35(36) + 35(9) = 
\\
= 4221 ways
\end{solution}

\newpage

% binomial coefficients and identities
\begin{problem} (10 points) Section 6.4, Exercise 8
\end{problem}
\begin{solution}:
\\
We want to find the coefficient of $x^8y^9$
\\
We are given $(3x+2y)^{17}$
\\
By looking at these two equations we can conclude that:
\\
n = 17
\\
j = 9
\\
$(3x+2y)^{17} = \Sigma_{j=0}^{17}$ $ {17}\choose{j} $ $3x^{17-j}2y^j$
\\
Plugging in j we get:
${17}\choose{9} $ $3x^{17-9}2y^9$
\\
= ${17}\choose{9} $ $3x^{8}2y^9$ 
\\
= ${17}\choose{9} $ $3^{8}2^9$ = $\frac{17!}{9!(8!)}$ ($3^82^9$)
\\
= using a calculator we can conclude that the final answer is 81662929920
\end{solution}

\newpage

% generalized permutations and combinations
\begin{problem} (10 points) Section 6, Supplementary Exercise 38
\end{problem}
\begin{solution}:
\\
Given the fact that we need to pick 12 dozens 3 of each kind of apples.
\\
There are 20 delicious apples in which we can pick 3 of the same kind.
\\
There are 20 macintosh apples in which we can pick 3 of the same kind.
\\ 
Finally, there are 20 granny smith apples in which we can pick 3 of the same kind.
\\
Therefore, 3$^3$ = 27 ways in which we choose dozen apples if at least three of each kind must be chosen.
\end{solution}

\newpage

% generating permutations and combinations
\begin{problem} (10 points) Section 6.6, Exercise 6
\end{problem}
\begin{solution}:
\\
Steps to complete this problem is as follows:
\\
1) look for LAST pair of integers where $a_j < a_{j+1}$
\\
2) look for the least integer to the right of $a_j$ that is greater than $a_j$
\\
3) now swap the value found in step 2 above and place it in the position $a_j$
\\
4) finally order the remaining elements
\\

A) 1342
\\
Using step 1 $a_2 = 3$ and $a_3 = 4$
\\
Using step 2 $a_3 = 4$
\\
Using step 3 $a_2 = 4$
\\
Using step 4 our answer is $1423$
\\

B) 45321
\\
Using step 1 $a_1 = 4$ and $a_2 = 5$
\\
Using step 2 $a_2 = 5$
\\
Using step 3 $a_1 = 5$
\\
Using step 4 our answer is $51234$
\\

C) 13245
\\
Using step 1 $a_4 = 4$ and $a_5 = 5$
\\
Using step 2 $a_5 = 5$
\\
Using step 3 $a_4 = 5$
\\
Using step 4 our answer is $13254$
\\

D) 612345
\\
Using step 1 $a_5 = 4$ and $a_6 = 5$
\\
Using step 2 $a_6 = 5$
\\
Using step 3 $a_5 = 5$
\\
Using step 4 our answer is $612354$
\\

E) 1623547
\\
Using step 1 $a_6 = 4$ and $a_7 = 7$
\\
Using step 2 $a_7 = 7$
\\
Using step 3 $a_6 = 7$
\\
Using step 4 our answer is $1623574$
\\

F) 23587416
\\
Using step 1 $a_7 = 1$ and $a_8 = 6$
\\
Using step 2 $a_8 = 6$
\\
Using step 3 $a_7 = 6$
\\
Using step 4 our answer is $23587461$
\\
\end{solution}

%\newpage

\goodbreak
\noindent
\textbf{Wildcard Quiz Problems} (the quiz on Friday could also be one of these)\\
Section 6, Supplementary Exercise 4\\
Section 6.2, Exercise 10\\
Section 6.3, Exercise 8\\
Section 6.5, Exercise 30\\



\goodbreak
\honor

\bigskip

\end{document}
