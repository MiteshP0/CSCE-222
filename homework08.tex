\documentclass{article}
\usepackage{amsmath,amsthm,amsfonts,latexsym,paralist}
\usepackage[margin=1in]{geometry}

\theoremstyle{definition}
\newtheorem{problem}{Problem}
\newtheorem*{solution}{Solution}
\newtheorem*{resources}{Resources}

\newcommand{\names}[3]{
\begin{center}
\begin{tabular}{|l|}
\hline
\textbf{Names of Group Members}\\
\hline
#1 \\
#2 \\
#3 \\
\hline
\end{tabular}
\end{center}
}
\newcommand{\honor}{\noindent \textbf{Aggie Honor Statement: }On my honor, as an Aggie, I have neither
  given nor received any unauthorized aid on any portion of the
  academic work included in this assignment. Furthermore, I have
  disclosed all resources (people, books, web sites, etc.) that have
  been used to prepare this homework. 
}

 
\newcommand{\checklist}{\noindent\textbf{Checklist:}
\begin{compactenum}
\item Did you type your full name and that of all collaborators? 
\item Did you abide by the Aggie Honor Code?
\item Did you solve all problems and start a new page for each? 
\item Did you submit your  PDF file?
\end{compactenum}
}

\newcommand{\problemset}[1]{\begin{center}\textbf{Problem Set #1}\end{center}}
\newcommand{\duedate}[1]{\begin{quote}\textbf{Due dates:} Electronic
    submission of \LaTeX\ and PDF files of this homework is due on
    \textbf{#1} on gradescope (\texttt{http://gradescope.com}). 
    \end{quote}}


\begin{document}
\begin{center}
{\large
CSCE 222 [505] Discrete Structures for Computing\\[.5ex]
Fall 2015 -- Philip C. Ritchey\\}
\end{center}

\problemset{1}

\duedate{11 November 2015 (Wednesday) before 11:30 a.m.}

\names{Mitesh Patel}
{UIN: 124002210}
{11-8-15}

\begin{resources} 
http://www.themathpage.com/aprecalc/mathematical-induction.htm \\
https://en.wikipedia.org/wiki/Equivalence\_relation\\
http://mathworld.wolfram.com/EquivalenceRelation.html \\
https://en.wikipedia.org/wiki/Equivalence\_class\\
http://mathworld.wolfram.com/EquivalenceClass.html\\
https://people.cs.pitt.edu/~milos/courses/cs441/lectures/Class16.pdf
\end{resources}

\bigskip

% Logic
\begin{problem} (10 points) \\
You have an 8-gallon bucket of water an empty 5-gallon bucket, and an empty 3-gallon bucket,  Prove or disprove that you can measure 4 gallons of water by successively pouring some or all of the water from one bucket into another bucket.
\end{problem}
\begin{solution} : \\
First, we can pour the 8 gallon bucket into the 5 gallon bucket so we will be left with 3 gallons of water in the 8 gallon bucket, a empty 3 gallon bucket, and a filled 5 gallon bucket. \\
Now we can pour the filled 5 gallon bucket into the empty 3 gallon bucket, giving us a filled 3 gallon bucket, and a 5 gallon bucket with 2 gallons of water. \\
Next lets pour the 3 gallon bucket into the 8 gallon bucket giving us 6 gallons of water in the 8 gallon bucket, a empty 3 gallon bucket, and a 5 gallon bucket with 2 gallons of water. \\
Now lets pour the 5 gallon bucket into the 3 gallon bucket giving us 2 gallons of water in the 3 gallon bucket, and a empty 5 gallon bucket. \\
Next pour the water in the 8 gallon bucket into the empty 5 gallon bucket and we get 1 gallon of water in the 8 gallon bucket, 5 gallons of water in the 5 gallon bucket, and 2 gallons of water in the 3 gallon bucket. \\
Finally, fill the 3 gallon bucket up with the 5 gallon bucket and we are left with 4 gallons of water in the 5 gallon bucket. Hence proved logically.

\end{solution}

\newpage

% Sequences
\begin{problem} (15 points)\\
Three numbers are in arithmetic progression.\\
Three other numbers are in geometric progression.\\
The corresponding terms of the two progressions sum to 85, 76, 84 respectively.\\
The sum of all three terms of the arithmetic progression is 126.\\
Find the terms of both progressions.
\end{problem}
\begin{solution} : \\
Form of Arithmetic progression: a, a+b, a+2b \\
Form of Geometric progression: c,cr,c$r^2$ \\

We know that :
\\
a+c=85 \\
a+b+cr=76 \\
a+2b+c$r^2$=84 \\

Solving system of equations:\\ 
a+0b+c=85 \\ 
- (a+b+cr) = 76 \\
----------------\\
-b+c(1-r) = 9\\

a+0b+cr=76 \\ 
- (a+2b+c$r^2$) = 84 \\
----------------\\
-b+cr(1-r) = -8\\

-b+c(1-r) = 9 \\ 
-(-b+cr(1-r)) = -8 \\
----------------\\
c(1-r) - cr(1-r) = 17 \\ 
= c($r^2$ - 2r + 1) = 17\\
= c$(r-1)^2$ = 17 \\
take r-1 = 1 \\
c = 17, and a = 68 by plugging in a+c=85 and solving for a\\
set r = 2: so therefore our answer will be : \\
Arithmetic: 68,42,16\\
Geometric: 17,34,68
\end{solution}

\newpage

% Big-O
\begin{problem} (15 points)\\
Show that $$\sum_{i=1}^n i^4 \log^2 i = \Theta(n^5 \log^2 n)$$
\end{problem}
\begin{solution} :
\\
Lets split and look at upper and lower bounds:
\\

$\theta$($n^4$)$\leq$ $i^4$ $\leq$ $\theta$($n^5$)
\\

$\theta$(log$^2$n)$\leq$ log$^2$i $\leq$ $\theta$(log$^2$n)
\\
Multiplying vertically we successfully shown:
\\

$\theta$($n^4$log$^2$n)$\leq$ i$^4$log$^2$i $\leq$ $\theta$(n$^5$log$^2$n)
\end{solution}

\newpage

% Induction
\begin{problem} (15 points) \\
A guest at a party is a celebrity if this person is known by every other guest, but knows none of them. There is at most one celebrity at a party, for if there were two, they would know each other. A particular party may have no celebrity. Your assignment is to find the celebrity, if one exists, at a party, by asking only one type of question -- asking a guest whether they know a second guest. Everyone must answer your questions truthfully. That is, if Alice and Bob are two people at the party, you can ask Alice whether she knows Bob; she must answer correctly. Use mathematical induction to show that if there are $n$ people at the party, then you can find the celebrity, if there is one, with $3(n-1)$ questions.\\
\textit{Hint: First ask a question to eliminate one person as a celebrity. Then use the inductive hypothesis to identify a potential celebrity. Finally, ask two more questions to determine whether that person is actually a celebrity.}
\end{problem}
\begin{solution} : \\

let s(n) be number of questions needed \\
p(n) is statement that s(n) $\leq$ 3(n-1) \\

Basis: n=1 the result is true\\
n=2 - p(2) if 2 people then we need only 2 questions to check if they are celebrity\\
s(2) = 2\\
$\leq$ 3(2-1) = 3\\
so proved.\\

Inductive : p(k) $\implies $ p(k+1) k $>$ 1
\\
Assume: p(k) if k people at party we find celebrity with 3(k-1) questions \\
show: p(k+1) - consider 2 people a and b, if a does not know b then b is not a celebrity, and if does b not know a then a is not celebrity. So we leave one person out. \\
Using IH from the k people left, we can find celebrity or none in atleast 3(k-1) questions. If none then result is true. \\

otherwise if c is celebrity then need 2 more questions to make sure c is actually a celebrity. Ask person left out if they know c then c is a celebrity, or if c knows the person then c is not a celebrity. \\
So max questions asked is \\
1+3(k-1) + 2 = 3k - 3 + 3 = 3k \\
if there is a celebrity we can find one in 3(n-1) questions for n $\geq$ 2. \\
We need only 3(n-1)-1 = 3n-4 questions so proved by PMI.
\end{solution}

\newpage

% Counting
\begin{problem} (15 points)\\
How many ways are there to travel in $xyzw$ space from the origin (0,0,0,0) to the point (8,6,7,5) by taking steps one unit in the positive $x$, positive $y$, positive $z$, or positive $w$ direction?
\end{problem}
\begin{solution} : \\

8+6+7+5 = 26 so 26 ways = 26!
\\
steps taken unit by unit in x y z and w direction so respectively 8! 6! 7! 5!.
\\
Therefore our answer is:
\\
Using c(n,r) formula:
\\
$\frac{26!}{8! 6! 7! 5!}$
\end{solution}

\newpage

% Divide and Conquer
\begin{problem} (15 points)\\
\begin{enumerate}
\item[(a)] Set up a divide-and-conquer recurrence relation for the number of modular multiplications required to compute $a^n \pmod{m}$, where $a$, $m$, and $n$ are positive integers, using the recursive algorithm from Example 4 in Section 5.4.
\item[(b)] Use the recurrence relation you found in part (a) to construct a big-O estimate for the number of modular multiplications used to compute $a^n \pmod{m}$ using the recursive algorithm.
\end{enumerate}
\end{problem}
\begin{solution} :
\\

A)\\
The recurrence relation is:
\\
T(n) = T(n/2) + 1 + 1 \\
T(n) = T(n/2) + 2 \\ 
T(n) = T(n/2) + O(1) \\
\\
This is because we are dividing into 1 subproblem each subproblem receiving 1/2 of the input. The time it takes to from these subproblems and combine their solution is O(1). O(1) because 1 + 1 indicates the exponent in the algorithm (We are multiplying).
\\

B)\\
We can construct big O by using master theorem:
\\
a = 1\\
b = 2 \\
d = 0 \\
b$^d$ = 1 \\

a = $b^d$  1=1 so: O(n$^d$log$_b$n) = O(n$^0$log$_2$n) = O(logn)
\end{solution}

\newpage

% Relations
\begin{problem} (15 points)\\
Let $R =\{(a,b) \in \mathbb{Z}^2 \mid |a| = |b|\}$. \\
Show that $R$ is an equivalence relation and specify the equivalence classes of $R$.
\end{problem}
\begin{solution} : \\

Reflexive: $\forall$$_a$ aRa \\
$|a|$ = $|a|$ so yes it is reflexive.\\

Symmetric: $\forall$$_{ab}$ \\
aRb $\implies$ bRa \\
$|a|$=$|b|$ $\implies$ $|b|$=$|a|$ so yes it is symmetric\\

Transitive:  $\forall$$_{abc}$ \\
(aRb $\wedge$ bRc) $\implies$ (aRc) \\
($|a|$=$|b|$ $\wedge$ $|b|$=$|c|$ ) $\implies$ ($|a|$=$|c|$) so yes it is transitive
\\

Therefore, since it is reflexive, symmetric, and transitive it is a equivalence relation.
\\

Specify equivalence classes:
\\
.[a] = \{b $|$ $|a|$ = $|b|$\}
\\
By example: lets consider set S = \{1,2,3,4,5\} = \{(1,1),(2,2),(3,3), (4,4), (5,5), (1,2), (2,1), (2,3), (3,2), (1,3), (3,1)\} \\

.[1] = \{1,2,3\} \\
.[2] = \{2,1,3\} \\
.[3] = \{3,2,1\} \\
.[4] = \{4\} \\
.[5] = \{5\} \\
\end{solution}

\newpage


\goodbreak
\honor

\bigskip
\checklist
\end{document}
