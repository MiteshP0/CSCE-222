\documentclass{article}
\usepackage{amsmath,amsthm,latexsym,paralist}
\usepackage{algorithm}
\usepackage{algpseudocode}

\theoremstyle{definition}
\newtheorem{problem}{Problem}
\newtheorem*{solution}{Solution}
\newtheorem*{resources}{Resources}

\newcommand{\names}[5]{
\begin{tabular}{|ll|}
\hline
\textbf{Name}  & \textbf{Problems}\\
\hline
#1 & 1--10\\
#2 & #3\\
#4 & #5\\
\hline
\end{tabular}}
\newcommand{\honor}{\noindent \textbf{Aggie Honor Statement: }On my honor, as an Aggie, I have neither
  given nor received any unauthorized aid on any portion of the
  academic work included in this assignment. Furthermore, I have
  disclosed all resources (people, books, web sites, etc.) that have
  been used to prepare this homework. 
}

 
\newcommand{\checklist}{\noindent\textbf{Checklist:}
\begin{compactenum}
\item Did you type your full name and UIN and those of any collaborators? 
\item Did you abide by the Aggie Honor Code?
\item Did you solve all problems and start a new page for each? 
\item Did you submit
\begin{compactenum}
\item your \LaTeX\ source file?
\item your  PDF file?
\end{compactenum}
\end{compactenum}
}

\newcommand{\problemset}[1]{\begin{center}\textbf{Problem Set #1}\end{center}}
\newcommand{\duedate}[1]{\begin{quote}\textbf{Due dates:} Electronic
    submission of \LaTeX\ and PDF files of this homework is due on
    \textbf{#1} on eCampus (\texttt{http://ecampus.tamu.edu}). 
    \end{quote}}


\begin{document}
\begin{center}
{\large
CSCE 222 [505] Discrete Structures for Computing\\[.5ex]
Fall 2015 -- Philip C. Ritchey\\}
\end{center}

\problemset{3}

\duedate{23 September 2015 (Wednesday) before 11:30 a.m.}

\names{Mitesh Patel}
{UIN: 124002210}{}
{9-22-15}{}

\begin{resources} Discrete Structures and its Applications - Rosen Chapter 3. \newline https://rob-bell.net/2009/06/a-beginners-guide-to-big-o-notation/ \newline http://www.trans4mind.com/personal-development/mathematics/series/sumNaturalSquares.htm \newline http://www.ee.ryerson.ca/~courses/coe428/sorting/insertionsort.html \newline http://codeyarns.com/2012/06/29/inserting-an-algorithm-in-latex/ \end{resources}

\bigskip

% describe an algorithm
\begin{problem} (10 points) Section 3.1, Exercise 4
\end{problem}
\begin{solution}:
\\

%procedure diff(x$_1$....x$_n$ : integers)
%\newline
%hold large = x$_1$ - x$_2$
%\newline
%for i := 2 to n-1
%\\
%\hspace{4ex} if hold large $<$ x$_{i+1}$ - x$_i$ then
%\\
%\hspace{4ex} hold large := x$_{i+1}$ - x$_i$
%\\
%return hold large
\begin{algorithm}
\caption{Find largest difference in a list of integers}
\label{}
\begin{algorithmic}[1]
\Procedure{diff}{$x_1 .... x_n : integers$}
    \State $large := x_1 - x_2$
    \For { i := 2 to n-1 }
         \If {$large < x_{i+1} - x_i$ } 
        large := $x_{i+1} - x_i$
        \EndIf
	\EndFor
	\State Return $large$
\EndProcedure
\end{algorithmic}
\end{algorithm}
\end{solution}

\newpage

% 1st and 2nd largest elements
\begin{problem} (10 points) Supplementary Exercise 2
\end{problem}
\begin{solution}:
\\ A) \\
\begin{algorithm}
\caption{Find first and second largest element in list of integers}
\label{}
\begin{algorithmic}[1]
\Procedure{find two}{$x_1 .... x_n : integers$}
    \State $first := x_1$
    \For { i := 2 to n }
         \If {$first < x_i$ } 
        first := $x_i$
        \EndIf
	\EndFor
	
	\State $second := x_1$
    	\For { i := 2 to n }
         \If {$second < x_i$ and second < first} 
        second := $x_i$
        \EndIf
	\EndFor
	\State Return $first$
	\State Return $second$
\EndProcedure
\end{algorithmic}
\end{algorithm}
\\
B) \\
O(n) for both for loops so number of comparisons is O(n).

\end{solution}

\newpage


% insertion sort
\begin{problem} (10 points) Section 3.1, Exercise 38
\end{problem}
\begin{solution}:
\\
Insertion Sort
\\
Original list: 6,  2,  3 , 1 , 5 , 4
\\
Checks: $ 6 > 2$
\\
After first check: 2,  6,  3 , 1 , 5 , 4
\\
Checks: $ 3 > 2, 3 > 6$
\\
After second check:  2,  3,  6 , 1 , 5 , 4
\\ 
Checks: $1 > 2, 1 <  3, 1 < 6$
\\
After third check: 1,  2,  3 , 6 , 5 , 4
\\
Checks: $5 > 1, 5 > 2, 5 > 3, 5 < 6$
\\
After fourth check: 1,  2,  3 , 5 , 6 , 4
\\
Checks: $4 > 1, 4 > 2, 4 > 3, 4 < 5, 4 < 6 $
\\
After fifth check: 1,  2,  3 , 4 , 5 , 6
\\ 
\\
List is now sorted after the fifth check.
\end{solution}

\newpage


% greedy algorithm
\begin{problem} (10 points) Section 3.1, Exercise 56
\end{problem}
\begin{solution}:
\\
We can use proof by counterexample:
\\ Our example could be : 20 cents.
\\ We first start with the normal way of counting change.
\\ 20 cents is two dimes which is two coins.
\\ Next, we use the 12 cent method to see how many coins it takes to satisfy 20 cents.
\\ Using the 12 cent method, we get 1 12 cent, 1 nickel, and 3 pennies which is 5 coins.
\\
\\
Therefore, it does not produce change using the fewest coins possible by our proof by counterexample. 
\end{solution}

\newpage


% big O
\begin{problem} (10 points) Supplementary Exercise 18
\end{problem}
\begin{solution}:
\\
$\Sigma_{j=1}^n$ j(j+1)
\\
= $\Sigma_{j=1}^n$ j$^2$ + $\Sigma_{j=1}^n$ j 
using: $\Sigma_{j=1}^n$ j  =  $\frac {n(n+1)} {2} $
\\
and $\Sigma_{j=1}^n$ j$^2$ = $\frac{n(n+1)(2n+1)}{6}$
\\
\\
= $\frac{n(n+1)(2n+1)}{6}$ + $\frac {n(n+1)} {2} $
\\
= $\frac {n(n+1)} {2} $ ($\frac{2n+1}{3}$ + 1)     by factoring
\\
= $\frac {n(n+1)(n+2)}{3}$
\\
So: f(n) = $\frac {n(n+1)(n+2)}{3}$ $\leq$ $2n^3$
\\
We know c = 2 and k =1 
\\ Therefore the Big O notation would be O($n^3$)
\end{solution}

\newpage


% ordering functions
\begin{problem} (10 points) Supplementary Exercise 26
\end{problem}
\begin{solution}:
\\
For this problem we need to know that n grows faster then log(n) and that log(log(n)) $<$ log(n). We can see this by graphing these functions on a graphing calculator.
\\
Looking at the growth rate of the functions given to order. We only care about the exponents so:
\\
$n^{log(n)} , nlog(n)loglog(n), n(log(n))^{3/2}, n^{4/3}log(n)^2, n^{3/2}, 2^{100n}, 2^{n^2}, 2^{2n}, 2^{n!}$
\end{solution}

\newpage


% big theta
\begin{problem} (10 points) Section 3.2, Exercise 30
\end{problem}
\begin{solution}:
\\
A) 3x+7 = O(x) b/c polynomial degree is 1
\\
x $\leq$ 3x
\\
x$\leq$ 3x+7
\\
c = 3 , k=1
\\
O(3x+7)
\\
3x+7 and x are same order 

\bigskip
B) $2x^2 + x - 7$ = O(x$^2$) b/c polynomial degree is 2
\\
$x^2 < 2x^2$
\\
$x^2 < 2x^2 + x - 7$
\\
c=2 k=7
\\
O($2x^2+x-7$)
\\
$2x^2+x-7$ and $x^2$ are same order

\bigskip
C) $[x+ \frac{1}{2}] $= O(x)
\\
$x\leq [x+ \frac{1}{2}]$
\\
c=1 k =1 
\\
O($[x+ \frac{1}{2}]$)
\\
$[x+ \frac{1}{2}]$ and x are same order
\\
\bigskip

D)$log(x^2+1)$ $\leq$ $log(2x^2)$
\\
$log(2x^2)$ = 2log(x)
\\
c = 2 ,  k=1
\\
= O(log(x))
\\
log(x) $\leq$ $log(x^2+1)$
\\
c=1 k=1
\\
O($log(x^2+1)$) 
\\ 
$log(x^2+1)$ and log(x) are same order
\bigskip

E)$log_{10}(x)$ $\leq$ log(x)
\\
O(log(x))
\\
log(x) $\leq$ $log_{10}(x)$
\\
O($log_{10}(x)$)
\\
$log_{10}(x)$ and log(x) are same order

\end{solution}

\newpage


% complexity
\begin{problem} (10 points) Supplementary Exercise 30
\end{problem}
\begin{solution}:
\\
A)
\\

\begin{algorithm}
\caption{sorts integers and checks for each pair of terms whether their difference is in the sequence}
\label{}
\begin{algorithmic}[1]
\Procedure{check}{$x_1 .... x_n : integers$}
    \State Sort ($x_1....x_n$) from high to low 
    \For { i := 1 to n-1 }
    	sub := $x_{i+1} - x_i$
	
		\For {k:= i+1 to n}
         	\If {$x_k$ := sub } 
        Return true
        \EndIf
	\EndFor
	\EndFor	
\EndProcedure
\end{algorithmic}
\end{algorithm}
\bigskip

B) O($n^2$) because nested for loops, and this algorithm is better than the brute force since it more efficient.
\end{solution}

\newpage


% complexity
\begin{problem} (10 points) Section 3.3, Exercise 16
\end{problem}
\begin{solution}:
\\
86400 seconds in one day which is approximately $10^5$
\\
so: $10^5 / 10^{-11}$ = $10^{16}$
\\
A) log(n) $\leq$ $10^{16}$
\\
n $\leq$ $2^{10^{16}}$
\\
$2^{10^{16}}$

\bigskip

B) 1000n $\leq$ $10^{16}$
\\
n  $\leq$ $10^{13}$
\\
$10^{13}$

\bigskip

C) $n^2$ $\leq$ $10^{16}$
\\
n $\leq$ $10^{8}$
\\
$10^{8}$
\bigskip

D) $1000n^2$ $\leq$ $10^{16}$
\\
$n^2$ $\leq$ $10^{13}$
\\
$n$ $\leq$ $10^{13/2}$
\\
$10^{13/2}$

\bigskip

E) $n^3$ $\leq$ $10^{16}$
\\
$10^{16/3}$

\bigskip

F) $2^n$ $\leq$ $10^{16}$
\\
n $\leq$ $log 10^{16}$
\\
$log 10^{16}$

\bigskip

G) $2^{2n}$ $\leq$ $10^{16}$
\\
n $\leq$ $\frac {log 10^{16}} {2}$
\\
 $\frac {log 10^{16}} {2}$
 \bigskip
 
 H) $2^{2^n}$ $\leq$ $10^{16}$
 \\
 $2^n$ $\leq$ $log 10^{16}$
 \\
 log($log 10^{16}$)
\end{solution}

\newpage


% complexity
\begin{problem} (10 points) Section 3.3, Exercise 46
\end{problem}
\begin{solution}:
\\
A)
\\
To use a brute-force algorithm we need to find a match for the first character of target then check each character afterwards. Using a while and a for loop to check for matching character of target and to check all the characters afterwards can help.
\\
\bigskip

B)
\\
\begin{algorithm}
\caption{Finding match for first character and checking successive characters for a match}
\label{}
\begin{algorithmic}[1]
\Procedure{match}{x: target , y: text}
    \State i := 1
    \While {$x_1 = y_i $ AND $  i \leq n$}
    TorF := 1
    \For { k := 1 to m }
    \If {$x_k$ != $y_{i+k}$}
    TorF := 0
    \EndIf
    \EndFor
    i := i+1
    \EndWhile
\EndProcedure
\end{algorithmic}
\end{algorithm}
\bigskip

C) Since outer loop runs n times and inner loop runs m times the big O is O(nm).
\end{solution}

%\newpage

\goodbreak


\goodbreak
\bigskip
\bigskip
\bigskip
\bigskip
\honor

\bigskip
\checklist
\end{document}
