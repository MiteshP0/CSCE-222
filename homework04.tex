\documentclass{article}
\usepackage{amsmath,amsthm,latexsym,paralist}

\theoremstyle{definition}
\newtheorem{problem}{Problem}
\newtheorem*{solution}{Solution}
\newtheorem*{resources}{Resources}

\newcommand{\names}[5]{
\begin{tabular}{|ll|}
\hline
\textbf{Name}  & \textbf{Problems}\\
\hline
#1 & 1--10\\
#2 & #3\\
#4 & #5\\
\hline
\end{tabular}}
\newcommand{\honor}{\noindent \textbf{Aggie Honor Statement: }On my honor, as an Aggie, I have neither
  given nor received any unauthorized aid on any portion of the
  academic work included in this assignment. Furthermore, I have
  disclosed all resources (people, books, web sites, etc.) that have
  been used to prepare this homework. 
}

 
\newcommand{\checklist}{\noindent\textbf{Checklist:}
\begin{compactenum}
\item Did you type your full name and UIN and those of any collaborators? 
\item Did you abide by the Aggie Honor Code?
\item Did you solve all problems and start a new page for each? 
\item Did you submit
\begin{compactenum}
\item your \LaTeX\ source file?
\item your  PDF file?
\end{compactenum}
\end{compactenum}
}

\newcommand{\problemset}[1]{\begin{center}\textbf{Problem Set #1}\end{center}}
\newcommand{\duedate}[1]{\begin{quote}\textbf{Due dates:} Electronic
    submission of \LaTeX\ and PDF files of this homework is due on
    \textbf{#1} on eCampus (\texttt{http://ecampus.tamu.edu}). 
    \end{quote}}


\begin{document}
\begin{center}
{\large
CSCE 222 [505] Discrete Structures for Computing\\[.5ex]
Fall 2015 -- Philip C. Ritchey\\}
\end{center}

\problemset{4}

\duedate{30 September 2015 (Wednesday) before 11:30 a.m.}

\names{Mitesh Patel}
{9-30-15}{}
{UIN: 124002210}{}

\begin{resources}Discrete Mathematics and its Applications by Rosen Chapter 5
 \\
http://math.stackexchange.com/questions/241060/structural-inductions
\\
http://highered.mheducation.com/sites/0073383090/student-view0/index.html
\\
https://www.youtube.com/watch?v=dMn5w4-ztSw
\\
https://www.youtube.com/watch?v=IFqna5F0kW8
\\
https://www.cs.cmu.edu/~adamchik/21-127/lectures/induction-1-print.pdf
\\
http://www.inf.ed.ac.uk/teaching/courses/dmmr/slides/13-14/Ch5.pdf
\\
http://math.stackexchange.com/questions/855680/discrete-math-induction-problem
\\
http://math.stackexchange.com/questions/517440/whats-the-difference-between-simple-induction-and-strong-induction
\end{resources}

\bigskip

% mathematical induction
\begin{problem} (10 points) Section 5.1, Exercise 4
\end{problem}
\begin{solution}:
\newline
p(n) = $1^3 + 2^3 + n^3 = (\frac {n(n+1)}{2} )^2$
\newline
A) $p(1) = 1^3 = (\frac{1(1+1)}{2})^2 $
\newline
B) Basis Step:$ 1^3 = (\frac{1(1+1)}{2})^2 $
\newline
1=1 so p(1) is True
\\
C) IH : Assume p(k) = $1^2 + ... + k^3 = (\frac{k(k+1)}{2})^2$
\\
D) Prove: $p(k) \to p(k+1)$
\\
Show: $ p(k) = 1^3 + ... + k^3 + (k+1)^3 = (\frac{k+1(k+2)}{2})^2$
\\
E) $ p(k) = k^3 = (\frac{k(k+1)}{2})^2$
\\
so:    $   k^3 + (k+1)^3 =  (\frac{k(k+1)}{2})^2 + (k+1)^3$   by IH
\\
= $(\frac{k(k+1)}{2}) (\frac{k(k+1)}{2}) + (k+1)^3 $
\\
= $(\frac{(k^2+1)(k^2+1)}{4}) + (k+1)^3$
\\ 
= $1/4 k^2 (k+1)^2 + (k+1)^2 + (k+1)$
\\
= $1/4 (k+1)^2(k^2+4k+4)$
\\
= $1/4(k+1)^2(k+2)^2$
\\
= $(\frac{k+1(k+2)}{2})^2$
\\ Therefore p(k+1) is true by PMI.
\\
F) The basis step is true and our inductive step is also true, so by PMI the formula is true whenever n is a positive number.
\end{solution}

\newpage

% mathematical induction + division
\begin{problem} (10 points) Supplementary Exercise 10
\end{problem}
\begin{solution}:
\\
$p(n) = n^3 + (n+1)^3 + (n+2)^3 divisible by 9$
\\
Basis: p(0) = 9 divisible by 9? YES so p(0) is true
\\
Inductive: p(k) $\to$ p(k+1)
\\ 
Assume p(k) = $ k^3 + (k+1)^3 + (k+2)^3$ divisible by 9
\\
Show p(k+1) is true :
\\
p(k+1) - p(k) using IH
\\
= $(k+1)^3 + (k+2)^3 + (k+3)^3 - k^3 - (k+1)^3 - (k+2)^3 $
\\
= $(k+3)^3 - k^3$
\\
= $(k^2+6k+9)(k+3) - k^2$ by expanding $(k+3)^3$
\\
= $9k^2 + 27k + 27$
\\
= $9(k^2+3k+3)$
\\
so we can see it is divisible by 9.
\\
therefore p(k) and p(k+1) are divisible by 9. 
\end{solution}
\newpage

% mathematical induction + recurrence *
\begin{problem} (10 points) Supplementary Exercise 20
\end{problem}
\begin{solution}:
\\
first few numbers of fibonacci:
\\
(0,1,1,2,3,5,8,13,21)
\\
Basis: f(0) = 0
\\
0 is divisible by of 3. means position 0 of the sequence which is the first element in the sequence. So:
\\
f(4) = 3 is a divisible of 3 so basis step is completed.
\\
Inductive: assume f(k) is divisible by 3
\\
show f(k+4) is divisible by 3
\\
f(k+4) = f(k+3) + f(k+2)
\\
= f(k+2) + f(k+1) + f(k+1) + f(k)
\\
= f(k+1) + f(k) + f(k+1) + f(k+1) + f(k)
\\
= 2f(k) + 3f(k+1)
\\
by IH f(k) is divisible by 3 and we know f(k+1) is some integer times 3 which is divisible by 3
\\
so f(k+4) is divisible by 3 by PMI
\end{solution}



\newpage

% strong induction - stamps walk through
\begin{problem} (10 points) Section 5.2, Exercise 4
\end{problem}
\begin{solution}:

A)Basis step:
\\
p(18) true b/c 1 4 cent stamp and 2 7 cent stamps
\\
p(19) true b/c 3 4 cent stamp and 1 7 cent stamp
\\
p(20) true b/c 5 4 cent stamps
\\
p(21) true b/c 3 7 cent stamps
\\
B) Inductive:  Assume p(i) for 18 $\leq$ i $\leq$ k
\\
C) Show (k+1) with 4 cent and 7 cent stamps
\\
D) (k+1)cent = (k-3)cent and 4 cents.. k-3+4 = k+1 so holds true
\\
E) True because both the basis and inductive steps were completed

\end{solution}

\newpage

% strong induction - find the flaw
\begin{problem} (10 points) Section 5.2, Exercise 32
\end{problem}
\begin{solution}:
\\
The flaw is in the induction step from the 4 cent case, we can form it using only 1 4 cent stamp so there is no need for 2 4 cent stamps.
\end{solution}

\newpage

% well-ordering
\begin{problem} (10 points) Supplementary Exercise 52
\end{problem}
\begin{solution}:
\\
A) No, not well ordered because it has infinite integers.
\\
B) Yes, it is well-ordered because least would be -99. 
\\
C) No, not well ordered because it has positive infinite rationals with no least element.
\\
D) Yes, it is well ordered because least element would be 1/100.
\end{solution}

\newpage

% recursive definitions + induction
\begin{problem} (10 points) Section 5.3, Exercise 13
\end{problem}
\begin{solution}:
\\
$p(n) = f_1 + f_3 +...+ f_{2n-1} = f_{2n}$
\\
Basis step : p(1) =$ f_1$ = 1 = $f_2$
\\
Inductive step: p(k) $\to$ p(k+1)
\\
Assume $p(k): f_1 + f_3 +...+f_{2k-1} = f_{2k}$
\\
Show p(k+1):  $f_1 + f_3 +...+ f_{2k-1} + f_{2k+1} = f_{2k} + f_{2k+1} $
\\
by IH  the right hand side =$ f_{2k} + f_{2k+1}$
\\ 
= $f_{2k+2}$
\\
= $f_{2(k+1)}$
\\
so:
\\
p(k+1) is true by PMI
\end{solution}

\newpage

% recursive definition + structural induction
\begin{problem} (10 points) Section 5.3, Exercise 32
\end{problem}
\begin{solution}:
\\
$\Sigma = (0,1)$
\\
x $\in \Sigma$ , t$\in \Sigma ^*$
\\
ones(tx) = ones(t) + x
\\
B) Base step: t = x
\\
ones(sx) = ones(s) + 0
\\
= ones(s) + ones(x) b/c ones(x) = 0 
\\
Assume ones(st) = ones(s) + ones(t)
\\
Recursive step: y $\in \Sigma$ s,t $\in \Sigma ^*$
\\
ones(s+y) = ones(st) + y
\\
= ones(s) + ones (t) + y b/c from above we get
\\
ones(s) + ones(ty)
\\
There fore by structural induction it is proved, because of our assumption we used in recursive step.

\end{solution}

\newpage

% structural induction + trees
\begin{problem} (10 points) Section 5.3, Exercise 44
\end{problem}
\begin{solution}:
\\
Basis step: since no internal vertices i(T) = 0
\\
so l(T) = 1
\\
= 1 + i(T)
\\
l(T) = 1+i(T) is True
\\
Assume : l(s) = 1 + i(S)
\\
Recursive step:
\\
If it is true for $T_1$ and $T_2$ then it is true for $T_1$
\\
know: $i(T) = i(T_1) + i(T_2) + 1$
\\
$l(T) = l(T_1) + l(T_2)$
\\
=$ i(T_1) + 1 + i(T_2) + 1$ from assumption
\\
= $i(T) + 1$ using $i(T) = i(T_1) + i(T_2) + 1$
\\
$l(T) = i(T) + 1$ is true by structural induction

\end{solution}

\newpage

% iterated functions
\begin{problem} (10 points) Section 5.3, Exercise 64
\end{problem}
\begin{solution}:
\\
f(n) = n/2 ... it is basically f divided by 2
\\
divide by $2^k$ to iterate through function so:
\\
$f^k (n)$ = $n/2^k$
\\
$f^k (n) \leq 1 $
\\
$n/2^k \leq 1$
\\
$n \leq 2^k$
\\
$log_2 (n) \leq k$
\\
so:
\\
$f_1^k (n) $ = $log_2 (n)$
\end{solution}

%\newpage

\goodbreak
\noindent




\goodbreak
\honor

\bigskip
\checklist
\end{document}
